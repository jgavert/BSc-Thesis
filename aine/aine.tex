% --- Template for thesis / report with tktltiki2 class ---

\documentclass[finnish]{tktltiki2}

% tktltiki2 automatically loads babel, so you can simply
% give the language parameter (e.g. finnish, swedish, english, british) as
% a parameter for the class: \documentclass[finnish]{tktltiki2}.
% The information on title and abstract is generated automatically depending on
% the language, see below if you need to change any of these manually.
%
% Class options:
% - grading                 -- Print labels for grading information on the front page.
% - disablelastpagecounter  -- Disables the automatic generation of page number information
%                              in the abstract. See also \numberofpagesinformation{} command below.
%
% The class also respects the following options of article class:
%   10pt, 11pt, 12pt, final, draft, oneside, twoside,
%   openright, openany, onecolumn, twocolumn, leqno, fleqn
%
% The default font size is 11pt. The paper size used is A4, other sizes are not supported.
%
% rubber: module pdftex

% --- General packages ---

\usepackage[utf8]{inputenc}
\usepackage{lmodern}
\usepackage{microtype}
\usepackage{amsfonts,amsmath,amssymb,amsthm,booktabs,color,enumitem,graphicx}
\usepackage[pdftex,hidelinks]{hyperref}

% Automatically set the PDF metadata fields
\makeatletter
\AtBeginDocument{\hypersetup{pdftitle = {\@title}, pdfauthor = {\@author}}}
\makeatother

% --- Language-related settings ---
%
% these should be modified according to your language

% babelbib for non-english bibliography using bibtex
\usepackage[fixlanguage]{babelbib}
\selectbiblanguage{finnish}

% add bibliography to the table of contents
\usepackage[nottoc,numbib]{tocbibind}
% tocbibind renames the bibliography, use the following to change it back
\settocbibname{Lähteet}

% --- Theorem environment definitions ---

\newtheorem{lau}{Lause}
\newtheorem{lem}[lau]{Lemma}
\newtheorem{kor}[lau]{Korollaari}

\theoremstyle{definition}
\newtheorem{maar}[lau]{Määritelmä}
\newtheorem{ong}{Ongelma}
\newtheorem{alg}[lau]{Algoritmi}
\newtheorem{esim}[lau]{Esimerkki}

\theoremstyle{remark}
\newtheorem*{huom}{Huomautus}


% --- tktltiki2 options ---
%
% The following commands define the information used to generate title and
% abstract pages. The following entries should be always specified:

\title{Välimuistiriippumattomat B-puut}
\author{Juho-Emil Antti Gävert}
\date{\today}
\level{Referaatti}
\abstract{Tiivistelmä.}

% The following can be used to specify keywords and classification of the paper:

\keywords{algoritmit, välimuistiriippumatonmalli, B-puut}
\classification{} % classification according to ACM Computing Classification System (http://www.acm.org/about/class/)
                  % This is probably mostly relevant for computer scientists

% If the automatic page number counting is not working as desired in your case,
% uncomment the following to manually set the number of pages displayed in the abstract page:
%
% \numberofpagesinformation{16 sivua + 10 sivua liitteissä}
%
% If you are not a computer scientist, you will want to uncomment the following by hand and specify
% your department, faculty and subject by hand:
%
% \faculty{Matemaattis-luonnontieteellinen}
% \department{Tietojenkäsittelytieteen laitos}
% \subject{Tietojenkäsittelytiede}
%
% If you are not from the University of Helsinki, then you will most likely want to set these also:
%
% \university{Helsingin Yliopisto}
% \universitylong{HELSINGIN YLIOPISTO --- HELSINGFORS UNIVERSITET --- UNIVERSITY OF HELSINKI} % displayed on the top of the abstract page
% \city{Helsinki}
%


\begin{document}

% --- Front matter ---

\maketitle        % title page
\makeabstract     % abstract page

% \tableofcontents  % table of contents
\newpage          % clear page after the table of contents


% --- Main matter ---

\section*{Välimuistiriippumattomat B-puut}

% Write some science here.

Michael A. Bender, Erik D. Demaine ja Martin Farach-Colton ovat kirjoittaneet
artikkelin 'Cache-Oblivious B-Trees', joka on julkastu artikkelina SIAM Journal
on Computing Vol 35. No 2. pp 341-358. Artikkelissa käsitellään
välimuistiriippumattomia tietorakenteita ja erityisesti miten he ovat saaneet
tehtyä kaksi välimuistiriippumatonta B-puuta.
\newline \indent Ohjelmoinnin opiskelun alussa opetellaan tietorakenteita.
Näissä tietorakenteissa puhutaan suoritusaika analyysistä ja
muistinkulutuksesta. Niissä on yleinen vika, joka johtuu tietokoneiden nopeasta
kehittymisestä ja materiaalin nopeasta vanhentumisesta. Ensimmäisissä
tietokoneissa ei ollut kuin välimuisti ja kovalevy. Muistihierarkia käsitti
koostui siis vain 2:sta tasosta. Nykyisissä koneissa on monia välimuistitasoja
joka edellyttää, että ne pitäisi ottaa huomioon tietorakenteiden
suunnittelussa. Tätä korostaa välimuistitasojen eri nopeudet. Siis välimuistien
tehokas käyttö on haastavaa. On olemassa välimuistimalleja, jotka yrittävät
ennakoida välimuistihierarkian vaikutuksen algoritmeihin, mutta nämä mallit
olettavat muistinkannalta pullonkaulan olevan tiedossa ja ne ovat yleensä tehty
jollekkin tietylle välimuistihierarkialle.
\newline \indent Ratkaisu ongelmaan on välimuistiriippumaton malli. Mallin teoria perustuu
ihanteelliseen välimuistimalliin. Mallissa käsitellään muistia kaksitasoisena
muistihierarkiana, mutta malli toimii mille tahansa muistihierarkialle. Frigo
ja Prokop esittelivät tämän mallin papereissaan. Niisä he esittelevät kuinka
ongelmille kuten, nopea Fourier muunnos ja matriisi kertolaskuun löytyy
optimaaliset algoritmit jotka ovat välimuistiriippumattomia. Nämä algoritmit
suorittavat asymptoottisesti optimaalisen määrän muistisiirtoja millä tahansa
muistihierarkialla ja kaikilla tasoilla hierarkiassa. Välimuistiriippumatonmalli
perustuu ideaan missä käsitellään 2-tasoista muistihierarkiaa. Päämuisti
on jaettu johonkin \(B\) määrään muistisoluihin ja välimuistilla on koko \(M\) ja se
voi säilyttää \(M/B\) määrän muistilohkoja. \(M/B\):n arvo on myös tarpeeksi suuri.
\(M\) ja \(B\) arvot ovat tuntemattomia algoritmille tai tietorakenteelle. Algoritmi
ei huolehdi muistista erityisesti. (Kun algoritmi yrittää päästä muistipaikkaan
joka ei ole välimuistissa niin tällöin järjestelmä hakee kyseisen muistilohkon
välimuistiin ja tätä kutsutaan muistisiirroksi. Jos välimuisti on täynnä niin
korvataan muistin tulevaisuuden käytön kannalta optimaalisesti valittu
muistilohko.) Frigo näyttää paperissaan, että näin voidaan simuloida mitä
tahansa muistihierarkiaa( pienellä vakiokerroin overheadilla). Siis voidaan
laskea välimuisti-riippumattomien mallien muistin käyttöä ihanteellisella
välimuistimallilla.
\newline \indent B-puun suunnittelu välimuisti-riippumattomaksi on hankalaa, koska puu
itsessään on suunniteltu kaksitasoiselle muistihierarkialle. Joko optimoiduksi
levylle kirjoituksia varten tai välimuistin kannalta. Tämän muuttaminen mille
tahansa muistihierarkialle optimoiduksi on haaste.
\newline \indent Välimuisti-riippumattoman mallin avulla voidaan ratkaista tämä ongelma.
Kehitettiin kaksi B-puuta seuraavilla muistisiirto rajoilla. Ensimmäinen puu sai
haku operaation muistirajoiksi B-puun hakurajan \(O(1 + \log_{B+1} N)\). Päivitys operaation rajaksi
saatiin sama kuin B-puiden rajan eli \(O(1 + \log_{B+1} N)\). Toisessa
hakupuussa lisätään skannaus operaatio. Puun haku operaatio on edelleen sama
kuin tyypillisen B-puun eli \(O(1 + \log_{B+1} N)\). Skannaus -operaatio saavutti optimaalisen
muistikäytön \(O(1+S/B)\). Päivitys operaatio sai rajan
\(O(1+\log_{B+1}N+\tfrac{\log^2 N}{B})\).
\newline \indent Välimuistiriippumaton puu koostuu monesta eri osasta, mutta
perusrakenteeltaan hyvin AVL-puuta vastaava. Puun
\(N\) määrä elementtejä on jaettu ryhmiin, joissa on \(O(\log N)\) elementtiä. Ryhmät on
tallennettu joukkoon, jonka koko on myös \(O(\log N)\). Jokainen elementti puussa
siis pitää sisällään pointterin sitä vastaavaan ryhmä joukkoon ja toisinpäin.
Mutta tämä johtaa siihen että etsiminen tässä puussa myös maksaa \(O(log N)\)
muistisiirtoja. Haluamme, että meillä on suhteellisen hitaat lisäykset
ja poistot ylimmässä osaa puuta, mutta meillä pitäisi olla nopeat etsinnät.
\newline \indent Yläpuun kompleksisuus halutaan pitää suurinpiirteisessä van Emde
Boas -sommitelmassa. Van Emde Boas -sommitelma on käytännössä muokattu versio
Prokopin sommitelmasta täydelliselle binääripuulle. Nimi tulee kuitenkin van
Emde Boas tietorakenteesta, jota se muistuttaa. Van Emde Boas -sommitelman
tärkeän ominaisuus on se, että millä tahansa yksityiskohtaisella tasolla
katsoessa, jokainen rekursiivinen alipuu on tallennettu jatkuvassa
muistilohkossa. Jotta voidaan pitää yläpuun kompleksisuus van Emde Boas
-sommitelmassa niin ensin käytetään paino-tasapainotettua B-puuta. Sen avulla
saadaan staattinen sommitelma muutettua dynaamiseksi. Käytetään avuksi dynaamisesti
tasapainotetun hakupuun kahta tärkeätä ominaisuutta. Ensimmäinen on
jälkeläisten jako pitääkseen elementit tasapainossa ja toiseksi tarvitaan
vahvaa tasapainotusta. Jollekkin vakiolle \(d\), jokaisessa elementissä \(v\)
korkeudessa \(h\) on \(O(d^h)\) jälkeläistä. Nämä kaksi ominaisuutta
löytyy mm. tasapainotetusta B-puusta. Hyödyntämällä tietämämme saatiin aikaiseksi
yläpuun haun muistisiirroksi \(O(1+\log_{B+1}N)\). Lisäyksien ja poistojen muistisiirto rajoiksi
saatiin \(O(\log^2N)\).
\newline \indent Keskiosa puuta tallentaa edustuselementit alemmasta osasta puuta, joista osa
voi olla kummitus elementtejä kuten myös keskitason elementeistä osa voi olla.
Myös yläosassa puuta voi olla kummituselementtejä. Keskitaso on implementoitu
yhtenä pakattuna muistirakenteena, missä edustus elementit toimivat ryhmien
erottimina muistissa. Alempipuu on taas joukko pakattuja muistirakenteita, jos
tarvitaan optimaalista skannaus operaatiota. Muuten se voi olla implementoitu
järjestämättöminä kokoelmina joukkoja. Näissä jokainen ryhmä on taas
tallennettu muistiin jatkuvana osana jossain järjestyksessä. Nämä alueet on
tallennettu muistiin saman kokoisina ja saa olla johonkin vakioon saakka tyhjänä.
Lisäykset ja poistot joukoista tarkoittaa, että koko joukko uudelleen
kirjoitetaan.
\newline \indent Lopullinen puu koostuu siis aiemmin mainituista osista. Saadaan
kaksi erilaista puuta riippuen tarpeesta, järjestetty tai järjestämätön B-puu.
Järjestetty tukee skannausta optimaalisesti. Molemmissa puissa haku operaatio vaatii
\(O(1+\log_{B+1}N)\) muistisiirtoa. Järjestetyn B-puun päivittäminen maksaa
\(O(1+\log_{B+1}N+\tfrac{\log^2 N}{B})\) muistisiirtoja. Missä itsessään päivitys
ilman hakua siis maksaa vain \(O(1+\tfrac{\log^2 N}{B})\). Järjestämättömässä B-puussa
päivitys maksaa \(O(1+\log_{B+1}N)\) amortisoituja muistisiirtoja. Nämä puut edustava ensimmäisiä dynaamisia
välimuistiriippumattomia B-puita.
\newline \indent On olemassa myös erilaisia lähestymistapoja välimuistiriippumattomiin B-puihin.
Yksi niistä perustuu ideaan bufferielementeistä. Käyttämättä epäsuoriaviittauksia
ja säilyttämällä puun pakatussamuisti joukossa. Idea perustuu että voidaan säilyttää
eri kokoista dataa yhdessä pakatussajoukossa. Tämä järjestetty puu sai seuraavat optimaaliset rajat:
haku \(O(\log_{B+1}N)\) muistisiirtoa, päivittäminen \(O(\log_{B+1}N)\) amortisoituja muistisiirtoja, Skannata \(S\) määrä elementtejä
vie \(O(1+S/B)\) amortisoituja muistisiirtoja.

% \cite{esimerkki}.


% --- Back matter ---
%
% bibtex is used to generate the bibliography. The babplain style
% will generate numeric references (e.g. [1]) appropriate for theoretical
% computer science. If you need alphanumeric references (e.g [Tur90]), use
%
% \bibliographystyle{babalpha}
%
% instead.

% \bibliographystyle{babplain}
% \bibliography{references-fi}


\end{document}
